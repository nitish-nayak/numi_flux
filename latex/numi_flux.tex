\documentclass[a4paper, oneside, 12pt]{article}

%\geometry{a4paper,top=2.5cm,bottom=4.6cm,left=1.5cm,right=1.5cm,columnsep=15pt,heightrounded}

\usepackage[T1]{fontenc}
\usepackage[utf8]{inputenc}
\usepackage[english]{babel}
%\usepackage[signatures]{frontespizio}
\usepackage{amsmath}
\usepackage{amsthm}
\usepackage{amssymb}
\usepackage{mathrsfs}
%\usepackage[bf,footnotesize]{caption}
\usepackage{booktabs}
\usepackage{graphicx}
\usepackage{caption}
\usepackage{subfig}
\usepackage[a4paper]{geometry}
\usepackage{indentfirst}
\usepackage{microtype}
\usepackage{sidecap}
\usepackage{bm}
\usepackage{braket}
\usepackage{slashed}
\usepackage{empheq}
\usepackage{epigraph}
\usepackage{xspace}
\usepackage{lscape}
\usepackage{xcolor}
\usepackage[output-decimal-marker={.}]{siunitx}
\usepackage{hhline}
\usepackage{comment}

\usepackage{feynmp-auto}
\unitlength=1mm

\usepackage{color}
\usepackage[colorlinks = true,
            linkcolor = red,
            urlcolor  = blue,
            citecolor = red,
            anchorcolor = blue]{hyperref}
\usepackage{makeidx}
\usepackage{listings}
\usepackage{color}
\usepackage{chngpage}
\usepackage{comment}
\usepackage{wrapfig}
\usepackage[autostyle,italian=guillemets]{csquotes}

\usepackage[sorting=none]{biblatex}
\bibliography{bibliography} % will look for bibliography.bib file

\newcommand{\mail}[1]{\href{mailto:#1}{\texttt{#1}}}
\newcommand{\Fluka}{\textsc{Fluka}\xspace}
\newcommand{\Flugg}{\textsc{Flugg}\xspace}
\newcommand{\G}{\textsc{G4NuMI}\xspace}
\newcommand{\Nova}{NO$\nu$A }
\newcommand{\dknu}{\texttt{dk2nu}\xspace}
\newcommand{\uB}{MicroBooNE\xspace}




\graphicspath{{images/}}





\begin{document}





\author{Marco Del Tutto}
\title{NuMI Flux at MicroBooNE}



\maketitle

\begin{abstract}
This document describes how the NuMI flux at MicroBooNE was evaluated for $\nu_\mu$, $\bar{\nu}_\mu$, $\nu_e$, $\bar{\nu}_e$ and shows plots of the flux in neutrino and anti-neutrino mode.
\end{abstract}


%\tableofcontents


\section{Flux Evaluation}

The NuMI neutrino beam is generated by colliding 120 GeV protons from the Fermilab main injector onto a 1.2 m graphite target. Two magnetic horns located downstream of the target focus charged particles of one sign along the beam direction and defocus charged particles of the opposite sign.

The simulation of the neutrino flux produced by the NuMI beam line is based on \textsc{Flugg} \cite{flugg} which uses the \textsc{Fluka} \cite{fluka} and \textsc{Geant4} \cite{geant} simulations. It includes a full simulation of the production of hadrons by the 120 GeV primary proton beam interacting with the NuMI target and the propagation of those hadrons through the target, magnetic horns, and along the decay pipe. The output is an \texttt{ntuple} representing the decays of secondaries (pions, kaons, etc.) that produce neutrinos.

\begin{comment}
The simulation software used is a combination of \textsc{Fluka} \cite{fluka} and \textsc{Geant4} \cite{geant}. The simulation of the physical processes is made by \textsc{Fluka} but the geometry description is written using \textsc{Geant4}. This combination is made by \textsc{Flugg} \cite{flugg}, a tool that add on to \textsc{Fluka} to interface to \textsc{Geant4} geometry. A simple model of incoming proton beam is used as a source of initial particles. The primary output of this code is an \texttt{ntuple} representing the decays of secondaries (pions, kaons, etc.) that give rise to neutrinos.
\end{comment}

%+++++++++FIGURE+++++++++
\begin{figure}[]
\centering
\includegraphics[width=1.0\textwidth]{circleDet}
\caption{Schematic representation of hadron decaying in a neutrino forced to a specific point of the detector.}
\captionsetup{format=hang,labelfont={sf,bf}}
\label{fig:circleDet}
\end{figure}
%++++++++++++++++++++++++


The decay of these ``neutrino parents'' is not simulated at this stage since we use a method to accelerate the conventional procedure of neutrino production, \cite{milburn}. The idea is to replace the Monte Carlo neutrino kinematics and transport to the detector with an analytic calculation based upon a rotation and Lorentz transformation of each small solid angle that points to the detector into the rest frame of the decaying hadron. Key to this calculation is the fact that the relevant hadrons are pseudoscalar and thus decay with the resulting neutrino emerging isotropically in this rest frame. The solid angle subtended by the detector element in this frame divided by $4\pi$ gives the probability that the neutrino will pass through the particular detector element. Each neutrino parent\footnote{Except for polarised muons, for which the decay angle w.r.t. spin direction is taken into account.} has then an associated weight
\[
w_\text{loc}=\frac{\Delta \Omega'}{4\pi},
\]
 where $\Delta \Omega'$ is the solid angle subtended by the detector element in the hadron rest frame, see Figure \ref{fig:circleDet}.



Since the solid angle in the laboratory frame is known and equal to
\[
\Delta \Omega = 2\pi (1-\cos\alpha),
\]
with $\alpha$ shown in Figure \ref{fig:circleDet}, one has to evaluate $\Delta \Omega'$. 
It is possible to show \cite{milburn} that $\Delta \Omega' = \Delta \Omega \cdot \text{M}^2$, where
\[
M = \frac{E}{E'} = \frac{1}{\gamma(1-\beta\cos\theta)},
\]
with $E$ and $E'$ being the energies of the decay neutrino in the laboratory and hadron rest frames respectively. $\theta$ is the polar angle of the neutrino momentum in the laboratory frame. $\beta$ is the usual $v/c$ and $\gamma$ is the Lorentz factor. 

One can get the neutrino flux looping over all the ``neutrino parents'' and applying a total weight given by
\[
w = w_\text{loc} \times w_\text{imp} \times w_\text{def},
\]
where:
\begin{itemize}
\item $w_\text{loc}$ is the location weight described above;
\item $w_\text{imp}$ is an importance weight that is already present in the \textsc{Flugg} file and has to be applied to each neutrino parent. This weight arises from the fact that, in order to speed up the beam simulation, not all the particles are propagated, but those that are propagated acquire a weight to account for the dropped ones;
\item $w_\text{def}$ is a default weight that allows us to get the flux per cm$^2$ and is equal to $1/10000\pi$.
\end{itemize}
 
After applying the above weight to each neutrino parent, the final NuMI neutrino flux energy spectrum can be obtained, as shown in Figure \ref{fig:numi_flux},  \ref{fig:numi_flux2}, \ref{fig:numi_flux_a} and \ref{fig:numi_flux_a2} decomposed w.r.t. $\nu_\mu$, $\bar{\nu}_\mu$, $\nu_e$, $\bar{\nu}_e$. The first two figures show the flux if NuMI is running in neutrino mode (forward horn current) and the last two if running in anti-neutrino mode (reverse horn current).

\section{Technical Information}

A software has been developed to perform this calculation. This software, called \texttt{NuMIFlux}, loops over all neutrino parents, picks up a random point in the \uB detector and calculates the appropriate location weight based on the formula above. The software is available on \texttt{github}: \url{https://github.com/marcodeltutto/NuMIFlux}.

The is a wiki page describing where the flux histogram files are and what they contain (so no need to run \texttt{NuMIFlux}): \url{https://cdcvs.fnal.gov/redmine/projects/ubooneoffline/wiki/NuMI_Flux_Histograms}.




\renewcommand*{\bibfont}{\footnotesize}
\nocite{*}
\printbibliography

\clearpage

%+++++++++FIGURE+++++++++
\begin{figure}[]
\centering
\includegraphics[width=0.8\textwidth]{numi_flux_numode_0-15}
\caption{NuMI neutrino flux at \uB with NuMI running in neutrino mode for four neutrinos: $\nu_\mu$, $\bar{\nu}_\mu$, $\nu_e$, $\bar{\nu}_e$. The distributions show the flux, not the event rate. Those are not cross-section weighted events.}
\captionsetup{format=hang,labelfont={sf,bf}}
\label{fig:numi_flux}
\end{figure}
%++++++++++++++++++++++++

%+++++++++FIGURE+++++++++
\begin{figure}[]
\centering
\includegraphics[width=0.8\textwidth]{numi_flux_numode_0-6}
\caption{NuMI neutrino flux at \uB with NuMI running in neutrino mode for four neutrinos: $\nu_\mu$, $\bar{\nu}_\mu$, $\nu_e$, $\bar{\nu}_e$. The distributions show the flux, not the event rate. Those are not cross-section weighted events. The only difference with Figure \ref{fig:numi_flux} is the $x$ range.}
\captionsetup{format=hang,labelfont={sf,bf}}
\label{fig:numi_flux2}
\end{figure}
%++++++++++++++++++++++++

%+++++++++FIGURE+++++++++
\begin{figure}[]
\centering
\includegraphics[width=0.8\textwidth]{numi_flux_antinumode_0-15}
\caption{NuMI neutrino flux at \uB with NuMI running in anti-neutrino mode for four neutrinos: $\nu_\mu$, $\bar{\nu}_\mu$, $\nu_e$, $\bar{\nu}_e$. The distributions show the flux, not the event rate. Those are not cross-section weighted events.}
\captionsetup{format=hang,labelfont={sf,bf}}
\label{fig:numi_flux_a}
\end{figure}
%++++++++++++++++++++++++

%+++++++++FIGURE+++++++++
\begin{figure}[]
\centering
\includegraphics[width=0.8\textwidth]{numi_flux_antinumode_0-6}
\caption{NuMI neutrino flux at \uB with NuMI running in anti-neutrino mode ffor four neutrinos: $\nu_\mu$, $\bar{\nu}_\mu$, $\nu_e$, $\bar{\nu}_e$. The distributions show the flux, not the event rate. Those are not cross-section weighted events. The only difference with Figure \ref{fig:numi_flux_a} is the $x$ range. }
\captionsetup{format=hang,labelfont={sf,bf}}
\label{fig:numi_flux_a2}
\end{figure}
%++++++++++++++++++++++++













\end{document}








